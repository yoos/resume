\documentclass[10pt,letterpaper,margin]{res}
% the margin option causes section titles to appear to the left of body text
\textwidth=6in % increase textwidth to get smaller right margin
\usepackage{fancyhdr}
\usepackage[utf8]{inputenc}
\usepackage{palatino}
\usepackage{microtype}
\usepackage{hyperref}
\usepackage{xcolor}

\renewcommand{\headrulewidth}{0pt}
\fancyhead{}
\fancyhead[L]{\emph{}}
\fancyhead[C]{\emph{}}
\fancyhead[R]{\emph{}}
\fancyfoot{}
\fancyfoot[C]{}
\pagestyle{fancy}

\hypersetup{colorlinks,
  linkcolor = black,
  citecolor = black,
  urlcolor  = black}
\urlstyle{same}

%\renewcommand{\labelitemi}{}

\definecolor{lightgray}{gray}{0.5}

\newcommand{\org}[2]{{\bf #1} \hfill {\color{lightgray} #2} \\}
\newcommand{\pos}[2]{\small {\it #1} \hfill {\color{lightgray} #2} \vspace{0.0em}}


\begin{document}

% Center the name over the entire width of resume:
\moveleft.5\hoffset\centerline{\huge\bf Soohyun Yoo}

% Draw a horizontal line the whole width of resume:
\moveleft\hoffset\vbox{\hrule width\resumewidth height 1pt}\smallskip

% Left column
% \moveleft.5\hoffset\centerline{iterativecode.com \hfill}
\moveleft.5\hoffset\centerline{github.com/yoos \hfill}
\moveleft.5\hoffset\centerline{\hfill}   % For alignment until the website's back up

% Center column
\vspace{-2.4em}
\moveleft.5\hoffset\centerline{yoos117@gmail.com}
\moveleft.5\hoffset\centerline{740-343-9667}

% Right column
\vspace{-2.4em}
\moveleft.5\hoffset\centerline{\hfill 22909 Fern Ave}
\moveleft.5\hoffset\centerline{\hfill Torrance, CA 90505}
\vspace{-0.2in}


\begin{resume}

\section{Education}

\org {Oregon State University} {Corvallis, OR}
\pos {B.S., Computer Science (3.38/4.0 GPA)} {Sep 2011 - Jun 2015}


\section{Work \\ Experience}

\org {SpaceX} {Hawthorne, CA}
\pos {Hardware Development Engineer II} {Apr 2018 -- Present}

\begin{itemize}
  \item Key member of the Lifecycle Engineering Team responsible for
    development of new tools and sustainable engineering processes. Heavy focus
    on nurturing effective use of JIRA and Confluence to reduce cost of context
    switching and facilitate knowledge sharing between teams.
  \item Shared team responsibilities include production support for all Dragon
    1 avionics and $>$90\% of Falcon 9 avionics, as well as providing
    surge support for other vehicle programs. Team of 10.
\end{itemize}


\pos {Hardware Development Engineer} {Dec 2015 -- Apr 2018}

\begin{itemize}
  \item Completed design and validation of:
    \begin{itemize}
      \item Four Xilinx UltraScale+ compute platforms built around a common
        "core block". Project kickoff to flight hardware validation completed
        in under one year. Co-designed with one other hardware design engineer,
        with tight coordination with procurement, manufacturing, software,
        FPGA, and program management teams.
      \item A general-purpose network switch, now part of Falcon, Dragon 2,
        Starship, and satellites
      \item A Secure GPS receiver, integrating a Selective Availability
        Anti-Spoofing Module (SAASM) provided by a third-party vendor, along
        with crypto key load procedures.
    \end{itemize}
  \item As a Responsible Engineer, led design reviews and completed multiple
    qualification campaigns (shock, vibration, thermal, EMC, and radiation
    testing).
\end{itemize}


\pos {Avionics Intern -- RIO Team} {Oct 2013 -- Mar 2014, Jul -- Dec 2015}

\begin{itemize}
  \item Developed software tools now used throughout Avionics to test anything
    with a processor and digital interfaces, with emphasis on performance and
    reliability.
  \item Supported confidence qualification of the Dragon 2 Flight Computer, as
    well as qualification of the Falcon Flight Computer leading up to the
    Falcon 9's return-to-flight launch and landing in Dec 2015.
  \item Implemented a PCIe Ethernet switch for radiation testing and evaluated
    an optoelectronic transceiver.
\end{itemize}


\org {OSU Robotic Decision Making Laboratory} {Corvallis, OR}
\pos {Research Assistant} {Jun 2014 -- Jun 2015}

\begin{itemize}
  \item Implemented an adaptive receding horizon explorer for an autonomous
    surface vehicle. Tested quadrotors in cooperative multiagent indoor
    exploration. Published to FSR and ICRA.
\end{itemize}


%%% Combined with SpaceX section above.
%\org {SpaceX} {Hawthorne, CA}
%\pos {Avionics Intern -- RIO Team} {Oct 2013 -- Mar 2014}
%
%\begin{itemize}
%  \item Implemented a PCIe Ethernet switch for radiation testing and evaluated
%    an optoelectronic transceiver.
%  \item Designed and built part of a software system for easy setup,
%    monitoring, and control of hardware test racks using ZeroMQ, AngularJS,
%    and CoffeeScript.
%  \item Designed and built a proof-of-concept automated derating tool in
%    Altium using DelphiScript.
%\end{itemize}


\org {OSU Open Source Laboratory} {Corvallis, OR}
\pos {Student Developer} {Feb -- Oct 2013}

\begin{itemize}
  \item Developed an administrative Python backend into an existing Drupal web
    framework and frontend for the Oregon Virtual School District. Team of 15.
\end{itemize}


\org {OSU Dynamic Robotics Laboratory} {Corvallis, OR}
\pos {Programmer} {Sep 2011 -- Oct 2013, Apr -- Jul 2014}

\begin{itemize}
  \item Assembled software system using ROS and Orocos for ATRIAS, an
    EtherCAT-enabled bipedal robot. System runs modular, hard real-time 1 kHz
    controllers in Xenomai-patched Linux, wirelessly controlled through a GTK+
    GUI with onboard and remote logging. Team of 5.
\end{itemize}

% \pos {Programming Intern} {Jun 2010 -- Aug 2010}
% 
% \begin{itemize}
%     \item Developed GTK+ GUI for ATRIAS 1.0, a monopod robot. Developed 3D
%         simulation and visualizer using Blender, Panda3D, and ODE. Software
%         used to test controllers.
% \end{itemize}



\section{Clubs \& \\ Activities}

\org {OSU Robotics Club Autonomous Aerial Team} {Corvallis, OR}
\pos {Team Lead, Programmer} {Sep 2011 -- Jun 2015}

\begin{itemize}
  \item Developed avionics, flight firmware using direction cosines for
    kinematic control, and worked with basic object recognition and third-party
    3D SLAM and navigation software. Organized budgets and schedule.
    % Team Lead Sep 2012 -- Oct 2013, Sep 2014 -- Jun 2015.
\end{itemize}



%\section{Personal \\ Projects}
%
%{\bf Tricopter} \hfill {\color{lightgray} Jan 2011 -- Aug 2012}\vspace{0.0em}
%
%\begin{itemize}
%    \item Multirotor with garage-built hardware and software. Served as
%        introduction to AVR microcontrollers, power electronics, cascading PID
%        control, ROS, and SolidWorks.
%\end{itemize}


% Consider removing this.
\section{Skills \& \\ Misc. Exp.}

{\bf Software:} C, C++, Python, Git, Linux (Arch, Ubuntu), JIRA/Confluence

{\bf Electrical:} High-speed digital design (Ethernet, SGMII, PCIe),
intermediate power supply design, basic RF \& mixed signals, Altium


% \section{References}
% 
% Available upon request.

\end{resume}



% \newpage
% \section{References}
% 
% {\bf Jonathan W. Hurst} \vspace{0.5em}\\
% Assistant Professor of Mechanical Engineering \\
% School of Mechanical, Industrial, and Manufacturing Engineering \\
% Oregon State University \vspace{0.5em}\\
% 218A Dearborn Hall \\
% Corvallis, OR 97331-6001 \vspace{0.5em}\\
% jonathan.hurst@oregonstate.edu \\
% 541-737-7010 \\
% 
% {\bf Cody Hyman} \vspace{0.5em}\\
% OSU Robotics Club President (2010 -- 2012) \vspace{0.5em}\\
% 770 NW 11th Street \\
% Corvallis, OR, 97330 \vspace{0.5em}\\
% hymanc@onid.orst.edu \\
% 541-350-3883\\

\end{document}

