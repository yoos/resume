\documentclass[10pt,letterpaper,margin]{res}
% the margin option causes section titles to appear to the left of body text
\textwidth=6.5in % increase textwidth to get smaller right margin
\usepackage{fancyhdr}
\usepackage[utf8]{inputenc}
\usepackage{palatino}
\usepackage{microtype}
\usepackage{hyperref}
\usepackage{xcolor}

\renewcommand{\headrulewidth}{0pt}
\fancyhead{}
\fancyhead[L]{\emph{}}
\fancyhead[C]{\emph{}}
\fancyhead[R]{\emph{}}
\fancyfoot{}
\fancyfoot[C]{}
\pagestyle{fancy}

\hypersetup{colorlinks,
  linkcolor = black,
  citecolor = black,
  urlcolor  = black}
\urlstyle{same}

%\renewcommand{\labelitemi}{}

\definecolor{lightgray}{gray}{0.5}

\newcommand{\cinfodiv}{{\color{lightgray} \hspace{1pt} -- \hspace{3pt}}}


\begin{document}

% Center the name over the entire width of resume:
 \moveleft.5\hoffset\centerline{\huge\bf Soo-Hyun Yoo}
% Draw a horizontal line the whole width of resume:
 \moveleft\hoffset\vbox{\hrule width\resumewidth height 1pt}\smallskip
% address begins here
% Again, the address lines must be centered over entire width of resume:
 \moveleft.5\hoffset\centerline{2201 NW Maser Drive, Corvallis, OR 97330}

% TODO: This is a hack to make my contact info look a bit more organized.
 \vspace{-1.2em}
 \moveleft.5\hoffset\centerline{yoos@onid.orst.edu \hfill iterativecode.com}
 \moveleft.5\hoffset\centerline{740-343-9667 \hfill github.com/yoos }

 \vspace{-0.1in}


\begin{resume}


\section{Education}

{\bf Oregon State University} \hfill {\color{lightgray} Corvallis, OR} \\
{\it B.S., Computer Science (3.25/4.0 GPA, projected graduation June 2015)} \hfill {\color{lightgray} Sep. 2011 -- Present}\vspace{0.2em}



% Skills section near top to impress HR people.
\section{Skills}

{\bf Languages}\vspace{0.2em}

\begin{itemize}
	\item Proficient: C, C++, Python, Bash, \LaTeX
	\item Basic: \hspace{17pt} HTML, CSS, Javascript, PHP
\end{itemize}

{\bf Software Tools \& Architectures}\vspace{0.2em}

\begin{itemize}
	\item Expert: \hspace{10pt} Vim, Git, Linux (Arch, Ubuntu)
	\item Proficient: ChibiOS embedded RTOS, Robot Operating System, Open Robot
		Control System
	\item Basic: \hspace{17pt} ZeroMQ, AngularJS, Qt, GTK+, Vagrant, Subversion, Mercurial
\end{itemize}

{\bf Mechanical \& Electrical}\vspace{0.2em}

\begin{itemize}
	\item High-speed digital design, AVR, ARM, Altium, Eagle PCB
%	\item Hand-soldering anything from 0402 SMDs to 0 gauge wire, PCB etching
	\item SolidWorks, MIG and arc welding, machine tools, aluminum anodization
\end{itemize}



\section{Work \\ Experience}

{\bf SpaceX} \hfill {\color{lightgray} Hawthorne, CA} \\
{\it Avionics Intern} \hfill {\color{lightgray} Oct. 2013 -- Present}\vspace{0.2em}

\begin{itemize}
	\item Designed testboards to:
		\begin{itemize}
			\item Evaluate optoelectronic transceivers for future use
			\item Test radiation susceptibility of a PCI-Express gigabit Ethernet switch IC
			\item Analyze efficiency of several switched-mode power supplies
			\item Measure S-parameters of a commonly used cPCI connector
		\end{itemize}
	\item Designed and conducted radiation tests of five devices at IU
		Cyclotron Facility.
	\item Currently designing and building a software system using ZeroMQ and
		AngularJS for easy setup, monitoring, and control of hardware test
		racks.
\end{itemize}


{\bf OSU Open Source Laboratory} \hfill {\color{lightgray} Corvallis, OR} \\
{\it Student Developer} \hfill {\color{lightgray} Feb. 2013 -- Oct. 2013}\vspace{0.2em}

\begin{itemize}
	\item Developed and integrated ORVSD administrative Python backend into
		existing web framework and user frontend interface in Drupal. Made
		small bugfixes in various other projects in a team of 15.
\end{itemize}


{\bf OSU Dynamic Robotics Laboratory} \hfill {\color{lightgray} Corvallis, OR} \\
{\it Programmer} \hfill {\color{lightgray} Sep. 2011 -- Oct. 2013}\vspace{0.2em}

\begin{itemize}
	\item Assembled software system in a team of five using ROS and Orocos for
		ATRIAS, an EtherCAT-enabled bipedal robot. System capable of running
		modular, hard real-time 1 kHz controllers in Xenomai-patched Linux,
		wirelessly controlled through a GTK+ GUI with onboard and remote
		logging.

% LEFT OUT TO KEEP SINGLE PAGE

%	\item Work within a team of four to create controller templates, allowing
%		researchers to write controllers for ATRIAS knowing only basic C++
%		syntax without the need to understand the entire software stack.
%	\item Run benchtop tests on commercial motor driver used by ATRIAS to
%		reverse engineer the proprietary current limiting algorithm and
%		determine the robot's overall performance cap.
%	\item Write Bash and Python scripts to automatically synchronize log files
%		between robot and GUI computers, convert log files to MATLAB format,
%		and deploy entire software system with a single command.
%	\item Help simulate the robot using Gazebo to troubleshoot potentially
%		destructive bugs in new controllers before running them on hardware.
%	\item Research high-performance IMUs for use in 3D walking. Work with
%		researchers at University of Michigan to integrate the KVH 1750 IMU
%		into ATRIAS's embedded controller firmware.
%	\item Help repair and assemble robot hardware and electronics.
\end{itemize}

{\it Programming Intern} \hfill {\color{lightgray} Jun. 2010 -- Aug. 2010}\vspace{0.2em}

\begin{itemize}
	\item Developed GTK+ GUI for ATRIAS 1.0, a monopod. Developed 3D simulation
		and visualizer of robot using Blender, Panda3D, and ODE. Software was
		used to test controllers before hardware completion.
\end{itemize}



\section{Clubs \& \\ Activities}

{\bf OSU Robotics Club Autonomous Aerial Team} \hfill {\color{lightgray} Corvallis, OR} \\
{\it Team Lead, Programmer} \hfill {\color{lightgray} Sep. 2011 -- Oct. 2013}\vspace{0.2em}

\begin{itemize}
	\item Developed embedded flight control firmware using direction cosine
		matrices for orientation kinematics. Developed software for object
		recognition and implemented third-party 3D SLAM and indoor navigation
		software. Organized budgets and yearly schedule. Team Lead Sep. 2012 --
		Oct. 2013.
\end{itemize}



\section{Personal \\ Projects}

{\bf Tricopter} \hfill {\color{lightgray} Jan. 2011 -- August 2012}\vspace{0.2em}

\begin{itemize}
	\item Three-bladed multirotor with garage-built hardware and software.
		Served as introduction to AVR microcontrollers, power electronics,
		cascading PID control, ROS, and SolidWorks.
% 	\item DIY PCB etching with cupric chloride
% 	\item DIY aluminum anodization
\end{itemize}

{\footnotesize (More personal projects on my website, iterativecode.com, and
Github page. References available upon request.)}


% \section{References}
% 
% Available upon request.

\end{resume}



% \newpage
% \section{References}
% 
% {\bf Jonathan W. Hurst} \vspace{0.5em}\\
% Assistant Professor of Mechanical Engineering \\
% School of Mechanical, Industrial, and Manufacturing Engineering \\
% Oregon State University \vspace{0.5em}\\
% 218A Dearborn Hall \\
% Corvallis, OR 97331-6001 \vspace{0.5em}\\
% jonathan.hurst@oregonstate.edu \\
% 541-737-7010 \\
% 
% {\bf Cody Hyman} \vspace{0.5em}\\
% OSU Robotics Club President (2010 -- 2012) \vspace{0.5em}\\
% 770 NW 11th Street \\
% Corvallis, OR, 97330 \vspace{0.5em}\\
% hymanc@onid.orst.edu \\
% 541-350-3883\\

\end{document}

