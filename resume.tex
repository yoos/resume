\documentclass[10pt,letterpaper,margin]{res}
% the margin option causes section titles to appear to the left of body text
\textwidth=6in % increase textwidth to get smaller right margin
\usepackage{fancyhdr}
\usepackage[utf8]{inputenc}
\usepackage{palatino}
\usepackage{microtype}
\usepackage{hyperref}
\usepackage{xcolor}

\renewcommand{\headrulewidth}{0pt}
\fancyhead{}
\fancyhead[L]{\emph{}}
\fancyhead[C]{\emph{}}
\fancyhead[R]{\emph{}}
\fancyfoot{}
\fancyfoot[C]{}
\pagestyle{fancy}

\hypersetup{colorlinks,
  linkcolor = black,
  citecolor = black,
  urlcolor  = black}
\urlstyle{same}

%\renewcommand{\labelitemi}{}

\definecolor{lightgray}{gray}{0.5}

\newcommand{\cinfodiv}{{\color{lightgray} \hspace{1pt} -- \hspace{3pt}}}


\begin{document}

% Center the name over the entire width of resume:
\moveleft.5\hoffset\centerline{\huge\bf Soo-Hyun Yoo}

% Draw a horizontal line the whole width of resume:
\moveleft\hoffset\vbox{\hrule width\resumewidth height 1pt}\smallskip

% Left column
\moveleft.5\hoffset\centerline{iterativecode.com \hfill}
\moveleft.5\hoffset\centerline{github.com/yoos \hfill}

% Center column
\vspace{-2.4em}
\moveleft.5\hoffset\centerline{yoos@onid.oregonstate.edu}
\moveleft.5\hoffset\centerline{740-343-9667}

% Right column
\vspace{-2.4em}
\moveleft.5\hoffset\centerline{\hfill 2201 NW Maser Drive}
\moveleft.5\hoffset\centerline{\hfill Corvallis, OR 97330}
\vspace{-0.1in}


\begin{resume}

\section{Education}

{\bf Oregon State University} \hfill {\color{lightgray} Corvallis, OR} \\
{\it Computer Science (3.42/4.0 GPA)} \hfill {\color{lightgray} (Expected) Sep 2011 -- Jun 2015}\vspace{0.0em}


\section{Work \\ Experience}

{\bf OSU Robotic Decision Making Laboratory} \hfill {\color{lightgray} Corvallis, OR} \\
{\it Research Assistant} \hfill {\color{lightgray} Jun 2014 -- Present}\vspace{0.0em}

\begin{itemize}
    \item Implement an adaptive receding horizon explorer for an autonomous
        surface vehicle. Help test quadrotors for cooperative multiagent indoor
        exploration.
\end{itemize}


{\bf SpaceX} \hfill {\color{lightgray} Hawthorne, CA} \\
{\it Avionics Intern -- RIO Team} \hfill {\color{lightgray} Oct 2013 -- Mar 2014}\vspace{0.0em}

\begin{itemize}
    \item Designed testboards and/or conducted tests to:
        \begin{itemize}
            \item Evaluate gigabit optoelectronic transceivers for future use
            \item Test a PCI-Express gigabit Ethernet switch IC for
                single-event effects in radiation
            \item Analyze efficiency of several switched-mode power supplies
            \item Measure S-parameters of a commonly used cPCI connector
        \end{itemize}
    \item Designed and conducted radiation tests of five devices at IU
        Cyclotron Facility.
    \item Designed and built part of a software system for easy setup,
        monitoring, and control of hardware test racks using ZeroMQ, AngularJS,
        and CoffeeScript.
    \item Designed and built a proof-of-concept automated derating tool in
        Altium using DelphiScript.
\end{itemize}


{\bf OSU Open Source Laboratory} \hfill {\color{lightgray} Corvallis, OR} \\
{\it Student Developer} \hfill {\color{lightgray} Feb -- Oct 2013}\vspace{0.0em}

\begin{itemize}
    \item Developed and integrated ORVSD administrative Python backend into
        existing web framework and frontend UI in Drupal. Made bugfixes in
        various other projects in a team of 15.
\end{itemize}


{\bf OSU Dynamic Robotics Laboratory} \hfill {\color{lightgray} Corvallis, OR} \\
{\it Programmer} \hfill {\color{lightgray} Sep 2011 -- Oct 2013, Apr -- Jul 2014}\vspace{0.0em}

\begin{itemize}
    \item Assembled software system in a team of five using ROS and Orocos for
        ATRIAS, an EtherCAT-enabled bipedal robot. System runs modular, hard
        real-time 1 kHz controllers in Xenomai-patched Linux, wirelessly
        controlled through a GTK+ GUI with onboard and remote logging.

% LEFT OUT TO KEEP SINGLE PAGE

%    \item Work within a team of four to create controller templates, allowing
%        researchers to write controllers for ATRIAS knowing only basic C++
%        syntax without the need to understand the entire software stack.
%    \item Run benchtop tests on commercial motor driver used by ATRIAS to
%        reverse engineer the proprietary current limiting algorithm and
%        determine the robot's overall performance cap.
%    \item Write Bash and Python scripts to automatically synchronize log files
%        between robot and GUI computers, convert log files to MATLAB format,
%        and deploy entire software system with a single command.
%    \item Help simulate the robot using Gazebo to troubleshoot potentially
%        destructive bugs in new controllers before running them on hardware.
%    \item Research high-performance IMUs for use in 3D walking. Work with
%        researchers at University of Michigan to integrate the KVH 1750 IMU
%        into ATRIAS's embedded controller firmware.
%    \item Help repair and assemble robot hardware and electronics.
\end{itemize}

{\it Programming Intern} \hfill {\color{lightgray} Jun 2010 -- Aug 2010}\vspace{0.0em}

\begin{itemize}
    \item Developed GTK+ GUI for ATRIAS 1.0, a monopod robot. Developed 3D
        simulation and visualizer using Blender, Panda3D, and ODE. Software
        used to test controllers.
\end{itemize}



\section{Clubs \& \\ Activities}

{\bf OSU Robotics Club Autonomous Aerial Team} \hfill {\color{lightgray} Corvallis, OR} \\
{\it Team Lead, Programmer} \hfill {\color{lightgray} Sep 2011 -- Oct 2013}\vspace{0.0em}

\begin{itemize}
    \item Developed embedded flight firmware using direction cosines for
        kinematic control. Developed software for basic object recognition and
        implemented third-party 3D SLAM and navigation software.  Organized
        budgets and schedule. Team Lead Sep 2012 -- Oct 2013, Sep 2014 --
        Present.
\end{itemize}



%\section{Personal \\ Projects}
%
%{\bf Tricopter} \hfill {\color{lightgray} Jan 2011 -- Aug 2012}\vspace{0.0em}
%
%\begin{itemize}
%    \item Multirotor with garage-built hardware and software. Served as
%        introduction to AVR microcontrollers, power electronics, cascading PID
%        control, ROS, and SolidWorks.
%\end{itemize}


% Consider removing this.
\section{Skills}

{\bf Software}\vspace{0.0em}

\begin{itemize}
    \item Expert: \hspace{10pt} Git, Linux (Arch, Ubuntu)
    \item Proficient: C, C++, Python, Bash, Robot Operating System, Open Robot Control System
    \item Basic: \hspace{17pt} OpenCV, ZeroMQ, JavaScript, DelphiScript
\end{itemize}

{\bf Electrical}\vspace{0.0em}

\begin{itemize}
    \item Proficient: High-speed digital logic design, Altium
\end{itemize}


\vspace{0.5em}
{\footnotesize (Personal projects on my website, iterativecode.com, and Github
page. References available upon request.)}

% \section{References}
% 
% Available upon request.

\end{resume}



% \newpage
% \section{References}
% 
% {\bf Jonathan W. Hurst} \vspace{0.5em}\\
% Assistant Professor of Mechanical Engineering \\
% School of Mechanical, Industrial, and Manufacturing Engineering \\
% Oregon State University \vspace{0.5em}\\
% 218A Dearborn Hall \\
% Corvallis, OR 97331-6001 \vspace{0.5em}\\
% jonathan.hurst@oregonstate.edu \\
% 541-737-7010 \\
% 
% {\bf Cody Hyman} \vspace{0.5em}\\
% OSU Robotics Club President (2010 -- 2012) \vspace{0.5em}\\
% 770 NW 11th Street \\
% Corvallis, OR, 97330 \vspace{0.5em}\\
% hymanc@onid.orst.edu \\
% 541-350-3883\\

\end{document}

