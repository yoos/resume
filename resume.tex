\documentclass[10pt,letterpaper,margin]{res}
% the margin option causes section titles to appear to the left of body text
\textwidth=6.5in % increase textwidth to get smaller right margin
\usepackage{fancyhdr}
\usepackage[utf8]{inputenc}
\usepackage{palatino}
\usepackage{microtype}
\usepackage{hyperref}
\usepackage{xcolor}

\renewcommand{\headrulewidth}{0pt}
\fancyhead{}
\fancyhead[L]{\emph{}}
\fancyhead[C]{\emph{}}
\fancyhead[R]{\emph{}}
\fancyfoot{}
\fancyfoot[C]{}
\pagestyle{fancy}

\hypersetup{colorlinks,
  linkcolor = black,
  citecolor = black,
  urlcolor  = black}
\urlstyle{same}

%\renewcommand{\labelitemi}{}

\definecolor{lightgray}{gray}{0.5}

\newcommand{\cinfodiv}{{\color{lightgray} \hspace{1pt} -- \hspace{3pt}}}


\begin{document}

% Center the name over the entire width of resume:
 \moveleft.5\hoffset\centerline{\huge\bf Soo-Hyun Yoo}
% Draw a horizontal line the whole width of resume:
 \moveleft\hoffset\vbox{\hrule width\resumewidth height 1pt}\smallskip
% address begins here
% Again, the address lines must be centered over entire width of resume:
 \moveleft.5\hoffset\centerline{2201 NW Maser Drive, Corvallis, OR 97330}

% TODO: This is a hack to make my contact info look a bit more organized.
 \vspace{-1.2em}
 \moveleft.5\hoffset\centerline{yoos@onid.orst.edu \hfill iterativecode.com}
 \moveleft.5\hoffset\centerline{740-343-9667 \hfill github.com/yoos }

 \vspace{-0.1in}


\begin{resume}


\section{Education}

Pursuing B.S. in Computer Science at Oregon State University, Corvallis, OR (3.25/4.0 GPA)

September 2011 -- Present (projected graduation June 2015)


% Skills section near top to impress HR people.
\section{Skills}

{\bf Languages}\vspace{0.2em}

\begin{itemize}
	\item Proficient: C, C++, Python, Bash, \LaTeX
	\item Basic: \hspace{17pt} HTML, CSS, Javascript, PHP
\end{itemize}

{\bf Software Tools \& Architectures}\vspace{0.2em}

\begin{itemize}
	\item Vim, Git, Subversion, Mercurial
	\item Linux (Arch, Ubuntu), AVR and ARM microprocessors, ChibiOS embedded RTOS
	\item Qt, GTK+, Robot Operating System (ROS), Open Robot Control System (Orocos)
\end{itemize}

{\bf Mechanical \& Electrical}\vspace{0.2em}

\begin{itemize}
	\item Hand-soldering anything from 0402 SMDs to 0 gauge wire, Eagle PCB, PCB etching
	\item SolidWorks, MIG and arc welding, machine tools, aluminum anodization
\end{itemize}



\section{Work \\ Experience}

% TODO: Get more meaningful experience before including this.
% {\bf OSU Open Source Laboratory} \hfill {\color{lightgray} Corvallis, OR} \\
% {\it Student Developer} \hfill {\color{lightgray} Feb. 2013 -- Present}\vspace{0.2em}
% 
% \begin{itemize}
% \end{itemize}


{\bf OSU Dynamic Robotics Laboratory} \hfill {\color{lightgray} Corvallis, OR} \\
{\it Programmer} \hfill {\color{lightgray} Sep. 2011 -- Present}\vspace{0.2em}

\begin{itemize}
	\item Build a software system for ATRIAS, a bipedal robot, that is capable
		of simultaneously running multiple modular, hard real-time 1 kHz
		controllers, wirelessly communicating with a graphical user interface,
		controlling six embedded EtherCAT slave devices, and reliably logging
		the full 1 kHz of data (approximately 10 MB/s) within Ubuntu Linux
		running on a mini-PC onboard the robot.
	\item Work within a team of four to create controller templates, allowing
		researchers to write controllers for ATRIAS knowing only basic C++
		syntax without the need to understand the entire software stack.
	\item Run benchtop tests on commercial motor driver used by ATRIAS to
		reverse engineer the proprietary current limiting algorithm and
		determine the robot's overall performance cap.
	\item Write Bash and Python scripts to automatically synchronize log files
		between robot and GUI computers, convert log files to MATLAB format,
		and deploy entire software system with a single command.
	\item Help simulate the robot using Gazebo to troubleshoot potentially
		destructive bugs in new controllers before running them on hardware.
	\item Research high-performance IMUs for use in 3D walking. Work with
		researchers at University of Michigan to integrate the KVH 1750 IMU
		into ATRIAS's embedded controller firmware.
%	\item Help repair and assemble robot hardware and electronics.
\end{itemize}

{\it Programming Intern} \hfill {\color{lightgray} Jun. 2010 -- Aug. 2010}\vspace{0.2em}

\begin{itemize}
	\item Developed graphical user interface in C++ and GTK+ for ATRIAS 1.0,
		a monopod.
	\item Developed a 3D simulation visualizer of the robot using Blender and
		Panda3D to be used to test hopping controllers before ATRIAS 1.0 was
		physically built.
\end{itemize}



\section{Clubs \& \\ Activities}

{\bf OSU Robotics Club Autonomous Aerial Team} \hfill {\color{lightgray} Corvallis, OR} \\
{\it Team Lead} \hfill {\color{lightgray} Sep. 2012 -- Present}\vspace{0.2em}

\begin{itemize}
	\item Coordinate technical developments and public outreach efforts.
\end{itemize}

{\it Programmer} \hfill {\color{lightgray} Sep. 2011 -- Present}\vspace{0.2em}

\begin{itemize}
	\item Develop embedded flight control algorithm using direction cosine
		matrices for orientation kinematics.
	\item Develop software for object recognition and autonomous indoor
		navigation with 3D localization and mapping towards competing in the
		International Aerial Robotics Competition.
\end{itemize}



\section{Personal \\ Projects}

{\bf Tricopter} \hfill {\color{lightgray} Jan. 2011 -- August 2012}\vspace{0.2em}

\begin{itemize}
	\item Three-bladed multirotor with garage-built hardware and software.
		Served as introduction to AVR microcontrollers, power electronics,
		cascading PID control, ROS, and SolidWorks.
% 	\item DIY PCB etching with cupric chloride
% 	\item DIY aluminum anodization
\end{itemize}

{\footnotesize (Please find more personal projects on my website, iterativecode.com, and Github page.)}



\section{References}

Available upon request.

\end{resume}



% \newpage
% \section{References}
% 
% {\bf Jonathan W. Hurst} \vspace{0.5em}\\
% Assistant Professor of Mechanical Engineering \\
% School of Mechanical, Industrial, and Manufacturing Engineering \\
% Oregon State University \vspace{0.5em}\\
% 218A Dearborn Hall \\
% Corvallis, OR 97331-6001 \vspace{0.5em}\\
% jonathan.hurst@oregonstate.edu \\
% 541-737-7010 \\
% 
% {\bf Cody Hyman} \vspace{0.5em}\\
% OSU Robotics Club President (2010 -- 2012) \vspace{0.5em}\\
% 770 NW 11th Street \\
% Corvallis, OR, 97330 \vspace{0.5em}\\
% hymanc@onid.orst.edu \\
% 541-350-3883\\
% 
% {\bf Will Rottenkolber} \vspace{0.5em}\\
% Crescent Valley High School Robotics Team Coach \vspace{0.5em}\\
% 4444 NW Highland Drive \\
% Corvallis, OR 97330 \vspace{0.5em}\\
% will.rottenkolber@gmail.com \\
% 541-231-2536 \\
% 
% {\bf Charles Creighton} \vspace{0.5em}\\
% Corvallis Youth Symphony Music Director \vspace{0.5em}\\
% P.O. Box 857 \\
% Corvallis, OR 97330 \vspace{0.5em}\\
% creighc@comcast.net \\

\end{document}

